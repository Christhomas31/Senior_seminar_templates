% This is a sample document using the University of Minnesota, Morris, Computer Science
% Senior Seminar modification of the ACM sig-alternate style. Much of this content is taken
% directly from the ACM sample document illustrating the use of the sig-alternate class. Certain
% parts that we never use have been removed to simplify the example, and a few additional
% components have been added.

% See https://github.com/UMM-CSci/Senior_seminar_templates for more info and to make
% suggestions and corrections.

\documentclass{sig-alternate}
\usepackage{color}

%%%% User-defined macros
\newcommand{\lam}{\lambda}
\newcommand{\mycomment}[1]{\textcolor{red}{#1}}
%%%%% Uncomment the following line and comment out the previous one
%%%%% to remove all comments
%%%%% NOTE: comments still occupy a line even if invisible;
%%%%% Don't write them as a separate paragraph
%\newcommand{\mycomment}[1]{}

\begin{document}

% --- Author Metadata here ---
%%% REMEMBER TO CHANGE THE SEMESTER AND YEAR
\conferenceinfo{UMM CSci Senior Seminar Conference, December 2013}{Morris, MN}

\title{An Overview of the Current State of Test-First vs. Test-Last Debate}

\numberofauthors{1}

\author{
% The command \alignauthor (no curly braces needed) should
% precede each author name, affiliation/snail-mail address and
% e-mail address. Additionally, tag each line of
% affiliation/address with \affaddr, and tag the
% e-mail address with \email.
\alignauthor
Chris M. Thomas\\
	\affaddr{Division of Science and Mathematics}\\
	\affaddr{University of Minnesota, Morris}\\
	\affaddr{Morris, Minnesota, USA 56267}\\
	\email{thom3706@morris.umn.edu}
}
\maketitle
\begin{abstract}

When it comes to software development, perhaps one of the most important and time consuming processes is that of software testing. In fact, early studies on software testing estimated that it could consume fifty percent or more of development costs for a product.  Because of this, the ability to optimize testing to reduce testing costs can be very valuable.  In this paper  we compare two popular methods, test-last testing, often used in waterfall software development processes, and test-first testing, often used in Agile test driven software development methods, by reviewing recent studies on the subject.  In this review we discuss the possible benefits of test-first and test-last testing and possible problems with the current data comparing these two testing methods.  After that, we explore other methods in test-first testing besides test driven development, such as behavior driven development, in an attempt to find a better test-first testing model.  In the end we discuss our results and potential future studies to help clarify current data.
\end{abstract}

% A category with the (minimum) three required fields

\keywords{ACM proceedings, \LaTeX, text tagging}

\section{Introduction}
When it comes to software development, perhaps one of the most important and time consuming processes is software testing.  In fact, some early studies on software testing estimated that it could consume fifty percent or more of the development costs for a product ~\cite{Bertolino:2007}.  Because of this, software developers have become increasingly interested in attempting to optimize testing to reduce development costs.

Although there are many testing methods that  exist, two particular testing methods are currently very popular: test-last testing and test-first testing.  Test-last testing, used mostly in process oriented or waterfall development, is a testing method where testing is done after software is written to ensure that the software is working as intended.  Test-first testing, used mostly in interval or agile oriented development, is a testing method where tests are written before the software being tested is written to ensure that the code to be written meets certain requirements.

Recently, there has been much debate in the testing community about whether or not test-first or test-last testing is superior.  The goal of this paper is to attempt to give an overview of the current state of this debate by analyzing current research data concerning the advantages and disadvantages of each testing method.  Because research often lags behind current implementations, and the field of test-first testing is currently changing due to its relatively new implementation, this paper will also explore new test-first methods.  This paper will discuss the advantages and disadvantages of test-first testing versus test-last testing and explore new test-first methodologies in an attempt to determine the applicability of each methodology.

The paper is divided into five sections.  In Section II we discuss what software testing is and current software development models with their supporting testing methods: test-first and test-last testing.  In Section III we will provide an analysis of the data explaining the potential advantages and disadvantages of test-first testing compared to test-last testing.  In Section IV we discuss issues of using test driven development to implement test-first testing. In Section V we will go over new testing methodologies based on test-first development. In Section VI we will provide conclusions and suggestions for further research in the field.

\section{Background}
\subsection{Software Testing}
Software testing, simply defined, is a branch of software engineering that uses a series of practices meant to either identify potential malfunctions or demonstrate functionality in a software system ~\cite{Bertolino:2007}.  Software testing can be as simple as running a program to see if its results look correct or can be as complex as writing code to simulate scenarios in the real world.

When comparing different testing methods there is no one standard quantitative measurement that determines which method is superior.  Because of this, many different types of measurement are used to argue that one testing method is better then another.  In this paper we will focus on three attributes that are commonly found throughout research: code coverage, total development time, and code correctness.   These three attributes are popular because they can be quantitatively measured and are considered important within the testing community.

Code coverage refers to the percentage of lines of production code that are executed when a set of tests are run. For example, if I write a set of tests that has 70\% test coverage, it means that 70\% of the lines of production code were executed when the tests were run.  Total development time refers to how long it takes to finish a development project.  Code correctness refers to how many errors are found within code after it is considered finished.  This is often measured by running a very large all encompassing test set against the participants code in the attempt to find cases where their code fails to produce the desired result. 

\subsection{Waterfall Development and Testing}
\mycomment{present tense this paragraph}
A popular software development model that was developed in the 1970's is the waterfall software development model, where software is developed in a series of phases.  The waterfall development model is popular because it is simple to implement correctly and is time efficient. The first phase in the standard waterfall model is the requirements phase, where requirements are set by the customer or design company.  Next the design phase occurs in which the product is designed. The product is then built in the implementation phase.  The final phase is the verification phase where testing and debugging occur ~\cite{wiki:xxx}.  The phases were set up in this manner to have previous phases make later phases simpler to complete.  Due to the fact that these processes often put testing at the end of development, a certain type of testing method, test-last testing, was the only testing method that made sense to use.
http://start.fedoraproject.org/
Test-last testing is currently a popular testing method and is usually the first testing method that people tend to implement.  Test-last testing is the practice of writing tests after code has been written to check the functionality of the written code.  These tests are then used by the developer to fix their code until no further errors are found by the tests.

\subsection{Agile Development and Testing}

In the late 1990s, a group of software developers started to criticize the phase-oriented waterfall model, complaining that it was too brittle and inflexible to meet the demands of the most customers.  In response to these critiques of the waterfall model, a new model for developing software emerged, the agile development model.  This new development model, based on the tenets of the agile manifesto ~\cite{agile:xxx}, promoted the idea that all actions of development should not occur within an ordered sequence of phases, but instead a series of time-boxed iterations where, in each iteration, developers set requirements, design a product, make the product, and test the product based on feedback from the previous iteration.  The goal of each iteration is to produce a demonstrable sub-product to show a customer and to receive feedback on that sub-product.  Some current development practices that are considered agile are Extreme Programming and Scrum.

Due to the changes in Agile programming, test-last testing was pushed aside in favor of a different style known as test-first testing.  In 2001, with the release of the agile development prachttp://start.fedoraproject.org/tice Extreme Programming, the idea of test-first testing, implemented in test driven development, started to become popular for the first time ~\cite{Hammond:2012}. Test-first testing is the practice of writing tests before code has been written and then writing code to make the tests pass.  It should be noted that since tests are written before production code, test-first testing tends to be heavily linked to development methods and thus the most common test-first models also include development elements as well. 

The most well known and used test-first model is that of test driven development, or TDD for short.  In the original TDD methodology, the developer uses a series of steps to develop his code.  The series begins once all current tests pass, or a new project is started.  When this occurs, the programmer writes a new failing test that tests the simplest functionality the programmer wishes to add to their code.  Once the test has been written and the test fails, the programmer then writes the minimal amount of code to make the test pass.  After this step the programmer streamlines his solution and integrates it with other parts of his code.  This series of steps repeats until the code is complete ~\cite{Hammond:2012}.  In its current use in the field, TDD no longer has the uniform meaning that is described above.  Instead it now refers to a loose collection of practices that roughly follow some or all of the guidelines given above.  This means that although there is a specific definition for TDD, we can not assume that a study used the formal definition of TDD unless the TDD process is outlined within the study. 

\section{Research Data on Test-First vs Test-Last Testing}
In this section we will attempt to obtain useful comparisons between test-first and test-last testing by reviewing current research articles. We will first review three main studies.  Next, we will summarize the results of our studies in terms of our three main empirical measurements: code coverage, code correctness, and total development time.  Afterwards we will discuss the potential issues with summarizing the data given in the research articles and then we will talk about potential useful comparisons we can draw from the data. 

\subsection{Data}

This subsection contains summaries of important studies that will be used later in the paper to draw out important conclusions and comparisons between test-first and test-last testing.  During this section we will make multiple references to test last development methods which we will refer to as TLD.

\subsubsection{Review by Kollanus}

First, is a 2010 review by Kollanus \cite{Kollanus:2010}.  In this review, Kollanus,  reviewed forty different experiments in scientific journals, magazines, and conference proceedings that provided empirical evidence comparing TDD  to TLD.  Each study was assigned to one of three categories: Controlled experiments, case studies, and others.  Controlled experiment articles were articles that went over a conducted controlled experiment.  Case study articles were articles that summarized received data from a group or situation over a period of time.  Other articles were any articles that did not fit into the above categories and were either non-controlled experiments or surveys.  Overall there were 14 controlled experiments, 14 case studies, and 12 studies defined as other.  


In this review, Kollanus focuses on three different code quality measurements: external code quality, productivity, and internal code quality.  External code quality, as defined by Kollanus, refers to how many errors are found in participants code.  This definition is the same as the definition for code correctness.  Because of this, we can use External code quality as measurement of code correctness.  Productivity, as defined by Kollanus, measures multiple properties related to how efficient the product code was to create.  This includes the measurement of total development time.  Although productivity is not strictly total development time, Kollanus wrote her conclusions in productivity based mainly on development time which suggests that the majority of productivity studies focused on total development time.  This means that the data given could be useful as a rough estimate of the effectiveness of total development time.
Internal code quality describes a wide set of measurements that measure code quality from a testing and development standpoint.  Code coverage is one of these measurements but is only mentioned briefly in a summary of one article.  Because of this, we are not able to assume that internal quality is a good estimate for code coverage.  Thus, we will be ignoring the overall results of internal code quality and focus on the summary of the one study.


 In the study, Kollanus concluded that there was weak support for improved external code quality in TDD methods compared to TLD methods.  This conclusion was based on the fact that out of the 22 studies that focused on external code quality, only 6 studies concluded that TDD did not increase external code quality.  However Kollanus is weary about this conclusion because out of the 7 controlled experiments that considered external code quality only 2 of them report an increase in code quality.  This was concerning to Kollanus because the data from the controlled experiment articles are generally more accurate than the other two types of articles.  
 
In terms of productivity, Kollanas suggests that TDD \textit{may} be less productive. This conclusion was drawn from 23 studies where 11 of the studies claimed that TDD decreases productivity, 7 of the studies say that there was no difference in productivity, and 5 studies that say there was an increase of productivity.  In this case the controlled experiment data accurately reflects these numbers as out of the 10 controlled experiments 2 of them claimed increased productivity, 4 of them claim no difference, and 4 of them claim decreased productivity.  It is interesting to note that although decreased productivity was the most common result in the review, the majority of the studies in the review state that TDD does not decrease productivity.
 
There was one study that Kollanus mentions that considered code coverage.  This mention occurs in the internal code quality section and noted that TDD improved test coverage.

\subsubsection{Review by Jeffries and Mitnik}

In 2012, Hellmann et. al conducted a summary on the current status of research of TDD.  Within Hellmann's background there is a mention of a summary study produced by Jeffries and Mitnik that found in general that TDD largely resulted in an increase in quality, but one study they identified showed instead that TDD resulted in a strong negative impact on quality.  Additionally while they showed that TDD could reduce the amount of effort required by up to 27\%, most studies found an increase in effort of up to 100\%.

\subsubsection{Experiment by Lemos et al} 

In 2012, Lemos et. al ~\cite{Lemos:2012} conducted a study on computer science students to see if test-first testing would significantly impact code coverage,  code correctness, and/or total development time in auxiliary functions (functions with 10-200 lines of code).   This study used 39 third-year computer science students knowledgeable in testing techniques. Each student took part in  two 100 minute test-first modules and were then asked to complete coding challenges over two sessions.  In the first session, half the students used test-first methods while the other half used test-last methods.  In the second session the students were asked to switch roles and given a different coding challenge.

Code coverage and total development time were measured as mentioned in Section II.  Code correctness was measured by running a set acceptance tests designed by the experimenter against a participant's code. Each code entry was then given a score based on how many acceptance tests passed.  The scores were either 0 (all test cases fail), 0.5 (some test cases fail) or 1 (all test cases pass).  This style of measuring code correctness has been used in previous studies and is called the \textit{Functional Test Set Success Level} scale.

It was found that code coverage on average was 40\% higher when test-first testing was used.  This percentage increase was found to be statistically significant. Thus, it was concluded that test-first methods produce higher test coverage then test-last methods. On total development time, it was found that test-first code took 12\% longer to write then test-last code.  This result was found to be significantly significant which lead to the conclusion that test-first methods took longer to implement then test-last methods. In terms of code correctness, the only difference found was that the test-last code had one more correct implementation of code then the test-first code.  In the case of test last code correctness, two submissions scored a 0, thirty-three submissions scored a 0.5, and five submissions scored a 1.  Test first code correctness scored slightly worse with two submissions that scored a 0, thirty-four submissions that scored a 0.5, and four submissions that scored a 1.  These results were not found to be statistically significant and thus it was concluded that test first testing had no impact on code quality compared to test last testing.
 
\subsubsection{Opinion Study by Janzen}

\subsection{Analysis}
Two of the three studies covered the topic of code coverage.  In both studies by Kollanus and Lemos et al. it was acknowledged that increased test coverage had occurred in test-first testing methods compared to their test-last counterparts.  All three papers brought up total development time and noted that test-first testing tended to increase development time, although both Kollanus's review and Hellmann's review provide notable contradictory studies to this conclusion.   All three studies in one way or another also brought up code correctness in the form of external code quality in Kollanus or in the form of code quality in the case of Hellmann et al.  In this case the results were across the board where Lemos et al. stated that there was no code correctness difference between test first and test last whereas Kollanus said there was weak evidence that test-first testing, in the form of TDD, produces more correct code then test-last testing.  Hellmann et al provided contradictory evidence backing that in most studies test-first testing has shown better code correctness then test last testing but there was a noted study where test-first testing was less code correct then test-last testing.   In summary, with the studies given, test-first testing produces more code coverage and is likely to take more time to code then test-last testing but due to contradictions in the research given we can not draw a conclusion on code correctness.

\subsection{Discussion}
Many summaries of TDD research ~\cite{Hammond:2012, Hellman:2012, Kettunen:2010} , including the literary review done by Kollanus ~\cite{Kollanus:2010}, have noted that it is hard to draw conclusions with the given research because that many studies that compare test-first to test-last testing are contradictory to one another.  For example in ~\cite{Hellman:2012},  there is a mention of previous summary studies that have had issues with conflicting data.  This is then followed by a series of examples of data that has conflicted in other studies in the past.  In Kollanus's literature review ~\cite{Kollanus:2010}, Kollanus points out that many articles in her study contradicted other articles which made it hard to make sound conclusions from the research given and may be a confounding factor in her study.  This seems to occur because of two main factors: participant experience and improper study implementation and documentation.

In the case of participant experience there seems to be a skill gap between test first and test last testing.  In an opinion study by ~\cite{Janzen:2007} it was found that a majority college students said they were not comfortable with and had trouble with comprehending TDD, even after doing a coding exercise with it.  The majority of students also said that they were much more comfortable implementing test last methods and would prefer to use test last methods in the future.  Coding professionals, on the other hand, actually preferred test first programming and were much more comfortable with it than students.  They also stated they felt it was relatively easy to understand while students did not. In another opinion study ~\cite{Kettunen:2010}, by Kettunen et al, a common theme throughout the paper was examples of individual testimonies and research conclusions that pointed to test first testing being harder to implement than test last testing.   Considering that most conducted experiments focus on college students and industry professionals, who have vast experience differences, the fact that test first testing may be harder to implement for college students compared to industry professionals may create differing results between studies.   This idea is considered in ~\cite{Hammond:2012} and ~\cite{Kettunen:2010}, but no summary papers where found considering this result and comparing student participant experiment results to professional participant experiment results.

The other problem that has plagued summary papers of test first testing is that most studies either lack documentation in key locations or do not effectively handle the participant conformance issue which comes up in TDD studies.  In the literary review by Kollanus, she acknowledges a frustrating lack of information in many studies on how TDD was implemented.  Most studies, claimed Kollanus, had maybe one or two lines describing their TDD methodology or just simply claimed that they used TDD.  Considering that TDD can be implemented in many ways, as discussed earlier in the background, Kollanus states that the lack of documentation for TDD implementation means that, as a reviewer, she has no idea if the test methods used to produce the results were similar or different, potentially confounding some of her conclusions.  Another issue similar to this is the conformance issue, brought up in (cite here).  The conformance issue in TDD testing refers to the fact that few researchers make sure that TDD was implemented correctly or at all by its participants.  For example, in the study by Lemos et. al the researchers acknowledged that one of their confounding factors was that they only asked students to write tests before code, thus a variety of different test first methods might have been used within the study making its data less conclusive.   Overall the lack of documentation of how researchers specifically planned to implement TDD in their study and their lack of some sort of monitoring of whether the implementation was actually occurring greatly reduces the credibility of the research being done and allows for confounding factors to potentially occur. 

\subsection{Conclusion}

Due to current issues in contradictory data it is hard to make solid conclusions about the advantages and disadvantages of using test first testing instead of test last testing.  What can be done instead is we can consider certain trends in research that can suggest potential answers.  Most studies and summaries seem to agree that test-first testing tends to produce more test coverage compared to test-last testing and that test first testing tends to take longer to write tests compared to test first testing.  Two of the studies we looked at, the experiment by lemos and the literary review by Kollanus, point out these two trends.  Also one or both trends have been noted through multiple papers ~\cite{Kettunen:2010, Hammond:2012, Hellman:2012} with no major contradictions found.  Another convincing trend that has occurred is that in general test first testing seems to be harder to implement than test last testing as mentioned earlier.   Although these trends seem to be clear, the rest seem to be more muddled.  Code correctness for example seems to have no clear trend as multiple studies have reached contradictory conclusions to other studies.  Another hot button topic that fails to show a trend one way or the other is the amount of time spent simply developing code as this again seems to have results across the board.  From these trends we can consider that test first testing overall probably takes more time to write, increases code coverage, and is harder to implement than test-last testing and that other advantages or detriments may exist but they are unclear in current research.

\section{Test Driven Development}
\mycomment{newly added section, still rough at best}

Test-first testing in most of these studies are being done in various forms of TDD.  Because of this, the data for test-first testing has the potential to reflect traits of TDD which are independent of any traits in test-first testing.  Although all the results have a potential to contain this issue one particular result seems to suggest it may be a trait of TDD, not test-first testing.  This trait is that Test-first testing is more difficult then test last testing.  

\subsection{Studies noting TDD difficulty}
Hammond et. al states in his research summary paper on TDD that: "TDD remains deceptively simple to describe but deeply challenging to put into practice effectively" `\cite{Hammond:2012}.  Hammond reaches this conclusion with the use of multiple studies that show implecations that TDD is complex.  Some of the studies used to draw this conclusion are summarized below.

In the study by Janzen 2007, there was an interesting relation between two different measurement aspects of the opinion study.  It was noted by Janzen that in every experiment, slightly less people said they would be more likely to implement TDD than people who said that TDD was a superior method to test-last methods. This means that more people thought that TDD was the better method to implement than  people who said they would actually implement TDD.  Janzen attributed this phenomena as a testament to the difficulty of TDD as he found in the comments section multiple survey participants mentioned that they felt that TDD was too difficult or too different from what they normally do.

In a study done by George and Williams,  \mycomment{cite this article!} multiple programming professionals from John Deere, RoleModel Software, and Ericsson participated in an experiment comparing TDD to waterfall development.  In this study a 9 question survey was given out asking programmers what they thought about TDD and what was difficult with TDD.  In this survey 56\% of the professionals noted that they had difficulty adapting to the TDD mindset when participating the study.  In addition 23\% of the participants noted that they felt that the lack of upfront design in TDD was more of a hindrance then a help. 

In an online survey done by Aniche and Gerosa ~\cite{Aniche:2010}, 218 TDD programmers of differing skill levels were surveyed about their TDD practices and mistakes.  In the survey it was found that TDD was not easy to follow as about 25\% of the programmers admitted to frequently or always making mistakes in following the traditional steps of TDD.  Two examples of these mistakes include: forgetting to clean up their code after a test passes and writing tests that are too complex for effective TDD.
 
\subsection{Reasons for TDD Difficulty}
The data above shows fairly strong evidence that TDD is at least somewhat difficult to implement but none of the data above shows that test-first testing is the problem.  In the first study, Janzen reveals that TDD was difficult for some people to implement.  In the second study, George and Williams reveal that 56\% of the participants found the TDD mindset hard to adapt while only 23\% of the participants found testing before designing as a problem, suggesting that something other then test-first testing caused difficulty for at least 33\% of the participants.  Whereas the third study shows that programmers have some trouble implementing the steps of TDD, which has nothing to do with testing before designing.  Since difficulty in using TDD does not seem to be directly linked with test-first testing, that suggests that there is different issue causing TDD to be difficult to implement.

One potential issue with TDD that is brought up by many developers is that it fails to explain the best way to implement tests to test your codes wanted functionality.  This is best represented in a quote from Dan North's article \mycomment{cite Introducing BDD here}: "While using and teaching agile practices like test-driven development (TDD) on projects in different environments, I kept coming across the same confusion and misunderstandings. Programmers wanted to know where to start, what to test and what not to test, how much to test in one go, what to call their tests, and how to understand why a test fails."  This issue as shown in Dan North's article may reveal why TDD is so difficult.  If the issue for why TDD is so difficult is similar to something like the issue explained above then perhaps they may be a better implementation other than TDD that exists.  

\section{evolutions to TDD}
\mycomment{this section is currently a work in progress}

As can been seen in the above section there is an issue where TDD can be difficult to implement correctly.  Due to this issue, new methods are starting to appear in the agile community that contain some of TDD's main tenets but have shifted enough away from TDD at some fundamental level that they are starting to receive their own names and classifications.  Although there are many of TDD spin-offs, for this paper we chose to focus on two more popular spin offs, acceptance test driven development and behavior driven development.
http://start.fedoraproject.org/
\mycomment{add lack of research disclaimer}

\subsection{Acceptance Test Driven Development}

Perhaps the closest spin off to TDD, acceptance test driven development (ATDD), follows many of the ideas and expectations that regular TDD does, like writing tests first and using tests to define the development code.  Unlike TDD, though, ATDD believes that in order to produce higher quality code, reduce coding confusion, and promote better customer satisfaction, customers should write or define "acceptance" tests.  These acceptance tests are tests that are required to pass before the customer will accept the product and usually define the core functionality of the product being developed.  This part of ATDD is noticeably different from the TDD ideology because in the TDD ideology only the developer should writes the tests and the tests should be written from simplest to hardest.~\cite{Hammond:2012}

The shift from having the developer write all the tests to the customer having to write some of tests is ideally considered a win win.  The shift of the workload of determining main functionality from the developer to the customer significantly reduces the difficulty of the TDD practice to the developer as they are no longer concerned about writing and determining all the functionality for their code.  This method is also supposed to benefit the customer as it allows the customer to get the product they desire.  One of the major complaints of ATDD is that it is an ideal practice at best because customers will not take the time to create useful acceptance tests.~\cite{Hammond:2012} Another complaint of ATDD is that its very hard for  customers to write useful acceptance tests  because they view functionality differently then developers because their knowledge of code is limited.(site new IEEE article here)  Because of these varied complaints, some of the promoters of ATDD have moved away from it to support a newer similar TDD spin off, Behavior Driven Development.

\subsection{Behavior Driven Development}
\mycomment{somewhat stubbed section atm}
Behavior Driven Development, or BDD, is a new spin off of TDD that focuses on how to correctly implement the fundamental usefulness of TDD.  The main difference between BDD and TDD is that TDD focuses on "testing the code" while BDD focuses more on defining a codes behavior. 
In order to do this, BDD focuses on defining a codes wanted functionality in plain English and then converts that plain English into tests using various programming tools and methods \mycomment{cite dan North and hammond article here}. This differs from TDD, because in TDD, their is no focus on defining code functionality in terms of plain English or even focusing specifically on code functionality.

The argued pros for BDD is that it can bypass many of the difficulties of TDD by allowing us to define the purpose of our code in more natural terms. This style removes much of the confusion about what is important to test in TDD and the other pro is that now it is easier to communicate goals and functionality of the code to non computer science majors (IE Managers, Customers, Sales Reps, etc.) which allows for higher customer satisfaction.

\section{conclusion}
\mycomment{this section is currently a work in progress}
\subsection{Conclusion intro}

In this paper we talked about two common test methods: test-first testing, often found in waterfall development models and test-last testing, often found in agile development models.  In section two we defined these test methods as well as provided background information to the reader.  In section three we explored recent research into the comparison of the two different testing methods and explored potential research problems with the current data.  In particular we found that data was contradictory but we still were able to tentatively conclude that test-first testing increases test coverage, increases time taken to test, and is harder to implement than test-last testing.  In this section we also suggested two potential reasons for the contradictory data: different participant experience levels and poor documentation/implementation of specific test first methods. In section four we explored different implementations of test first testing.  

\subsection{Discussi or BDDon of Implications of Section 3}
\mycomment{stubbed for now}

\subsection{Discussion of Implications of Section 4}
\mycomment{stubbed for now}
\subsection{Suggestion for Further Research}
For people who are looking to advance the field of TDD it would be potentially very interesting to see how the new TDD spin-offs such as Behavior Driven Development and Agile Specification-Driven Development compare to that of a well documented TDD process like the one suggested for Extreme Programming.  Another useful study that could be implemented would be to compare the results of studies thing that could also help advance the field is to do research comparing the effectiveness of waterfall testing and TDD testing between groups with notably different experience levels as some research tentatively points that this may be a relevant variable.

\mycomment{This paragraph is a rough idea on the further research topic}

\mycomment{conclusion brain storm}

-TDD increases code test coverage

-TDD perhaps takes more skill to preform

-TDD may have more effective and less effective practices

-More study is needed.

-Better study methods should be implimented

\bibliographystyle{abbrv}
% sample_paper.bib is the name of the BibTex file containing the
% bibliography entries. Note that you *don't* include the .bib ending here.
\bibliography{Paper_Draft}  
% You must have a proper ".bib" file
%  and remember to run:
% latex bibtex latex latex
% to resolve all references

\end{document}
